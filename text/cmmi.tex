\section{Capability Maturity Model Integrated}

Bei Capability Maturity Model Integrated (\textit{CMMI}) handelt es sich um ein Modell, welches sich eines Problems annimmt, welches etwa beim \textit{V-Modell} zu kurz kommt. Dort wird die Qualität lediglich auf der Projektebene betrachtet und somit kann keine generalisierbare Aussage über die Qualität getroffen werden. In Zeiten, in denen Entwicklungsprojekte häufig zu Fremdfirmen ausgelagert werden, ist es für Auftraggeber ein wichtiges Anliegen, geeignete Zulieferer von ungeeigneten zu unterscheiden. Die über \textit{CMMI} nachgewiesenen \textit{Reifegrade} stellen eine Möglichkeit dar, dieses Bedürfnis zu befriedigen.

Das Modell geht zurück auf \textit{CMM}, welches fünf Stufen bereitstellt, die den Reifegrad einer Organisation beschreiben.

\begin{itemize}
    \item Auf \textbf{Stufe 1: Anfänglich} beginnt jede Organisation, da es hier keine Anforderungen gibt. Auf dieser Stufe gibt es keinen geregelten Prozess und die Entwicklung kann als \textit{explorativ} beschrieben werden.
    \item \textbf{Stufe 2: Wiederholbar} erfordert, dass die Organisation eine Form von Projektmanagement anwendet und damit ein konstanter und wiederverwendbarer Prozess implementiert wird.
    \item Um die \textbf{Stufe 3: Definiert} zu erreichen, muss die Organisation ihren Prozess formal definieren.
    \item \textbf{Stufe 4: Gemanaged} wird erreicht, sobald die Organisation ihren definierten Prozess lenkt und zur Entscheidungsfindung spezifische Metriken einsetzt.
    \item Die höchste \textbf{Stufe 5: Optimierend} erfordert zuletzt, dass ein kontinuierlicher Verbesserungsprozess angewandt wird und dass mithilfe von Statistiken die Effizienz des Prozesses nachgewiesen wird.
\end{itemize}

Bei \textit{CMMI} handelt es sich nun um eine Erweiterung von \textit{CMM}. Hier kann eine Organisation verschiedene \textit{Maturity Levels (ML)} erreichen. Jedes bringt dabei verschiedene, konkretere Anforderungen mit sich. Es wird in Form von \textit{Capability Levels (CL)} gemessen, wie gut diese Anforderungen erfüllt werden. Sind definierte Schwellenwerte erreicht, arbeitet die Organisation auf dem entsprechendenn \textit{ML}.

Zur Zertifizierung gibt es ein weiteres Verfahren, das \textit{SCAMPI} genannt wird. Auch hier werden verschiedene Klassen definiert. Diese geben die Intensität und damit die Aussagekraft der Prüfung der Organisation an. Dabei bezeichnet die Klasse \textit{A} den höchsten Testumfang und die Klasse \textit{C} den geringsten.

Es lässt sich ein Zusammenhang zwischen dem Reifegrad und der Risikoaversion einer Organisation. Dementsprechend wird ein hoher Reifegrad oft dadurch erreicht, dass Projekte mit einem höheren Risiko nicht umgesetzt werden. Dies schmälert zum einen den Profit und zum anderen die Innovationskraft, da beides besonders in risikobehafteten Projekten gefördert wird.