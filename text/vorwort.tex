\section{Vorwort}
Diese Hausarbeit bearbeitet die folgenden Aufgaben in einer leicht veränderten Struktur. Aufgaben, in welchen die Modelle isoliert betrachtet werden, wurden zusammengefasst, während die verbleibenden in ein separates Kapitel übernommen wurden.

\begin{enumerate}
    \item Kurze Beschreibung der vorgestellten klassischen Modelle.
    \item Strukturierung und Einordnung der bisher kennengelernten Modelle in den Ordnungsrahmen der HELENA-Studie (siehe Paper "Hybrid Software and System Development in Practice: Waterfall, Scrum, and Beyond", z.B. Figure 2: Method oder Practice oder Table 7: Clustering der Techniken...). 
    \item Zu welcher Kategorie gehören die besprochenen Modelle jeweils?
    \item Welche Modelle überschreiten die im HELENA-Paper beschriebenen Entwicklungsansätze?
    \item Identifizierung und Vergleich der Bedeutung der verschiedenen Ansätze.
    \item Frage: Wie würden Sie sparsamen und ungeduldigen Kund*innen erklären,  warum sie höhere Aufwände verrechnen, weil  Sie zertifiziert sind (egal nach welcher Methodik – CMMI, ISO 15504. ISO 33000) und welche Vorteile es für ihn/sie hat.
\end{enumerate}

Die Fragen 1 bis 4 sind als \textit{modellspezifisch} eingestuft worden, entsprechend werden diese nach Modell gruppiert beantwortet. Frage 3 wurde dabei als ein Teil der Frage 2 aufgefasst. Die Fragen 5 und 6 erhalten jeweils ein eigenes Kapitel.

\newpage
\tableofcontents