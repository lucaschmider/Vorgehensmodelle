\section{Identifizierung und Vergleich der Bedeutung der verschiedenen Ansätze}
Wie die nachfolgende Liste zeigt, weichen die Bedeutungen der vorgestellten Ansätze stark von einander ab.

\begin{itemize}
    \item \textbf{CMMI} wird verwendet, um den richtigen Dienstleister für einen Entwicklungsauftrag zu finden
    \item Das \textbf{V-Modell} wird verwendet, wenn es sein muss. Zum Beispiel im öffentlichen Sektor oder der Pharmaindustrie
    \item Der \textbf{Unified Process} wird kaum verwendet. Vermutlich handelt es sich um eines der ersten Schritte in die Richtung der Agilität.
    \item \textbf{Kostenschätzungsmodelle} werden bei größeren Projekten verwendet, um zu entscheiden, ob das Projekt tatsächlich die Umsetzung wert ist.
\end{itemize}

Ein Vergleich im eigentlichen Sinne ist an dieser Stelle schwer möglich, da man Äpfel mit Birnen vergleichen würde. Die Bedeutung der Modelle lässt sich jedoch detaillierter beschreiben.

\paragraph{CMMI} Das Modell findet seine Bedeutung in Projekten jeder Größe. Um es sinnvoll einzusetzen, muss ein Unternehmen regelmäßig Entwicklungsaufträge von anderen Organisationen bearbeiten. Eine entsprechende Zertifizierung kann von der Organisation dann als ein Verkaufsargument genutzt werden. Bei einem hohen \textit{CMMI}-Level kann im Besonderen die Preisgestaltung gerechtfertigt werden.

\paragraph{V-Modell} Dieses Modell bringt durch den hohen Vorbereitungs- und Dokumentationgrad grundsätzlich hohe Kosten mit sich. In der Regel sind Unternehmen genau dann bereit, diese zu bezahlen, wenn sie es müssen. Konkret ist das bei Rüstungs- oder Pharmafirmen der Fall, wo die zuständigen Behörden ein entsprechendes Sicherheitsniveau verlangen.

\paragraph{Kostenschätzungsmodelle} Die beiden Modelle dieser Kategorie sind besonders bei größeren Projekten sinnvoll. Groß bezieht sich dabei insbesondere auf das Investitionsvolumen, dass geschätzt werden kann. Durch die detailliertere Auseinandersetzung mit den Anforderungen und den verbundenen Kosten können die Verantwortlichen sich noch einmal überlegen, ob das Projekt tatsächlich umgesetzt werden sollte. Ist dem so, können die Ergebnisse bei der Nutzung anderer Vorgehensmodelle helfen, da die Use-Cases bereits sehr nahe an den User-Stories sind, die etwa in Scrum benötigt werden.