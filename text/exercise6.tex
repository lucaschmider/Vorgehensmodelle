\section{Wie würden Sie einem sparsamen und ungeduldigen Kunden erklären, warum sie höhere Aufwände verrechnen, weil Sie zertifiziert sind (egal nach welcher Methodik – CMMI, ISO 15504. ISO 33000) und welche Vorteile es für ihn hat?}
Das Ziel der genannten Zertifizierungen ist es stets, einen erfolgreichen Projektabschluss zu garantieren. Andere Zertifizierungen belegen in der Regel, dass das Unternehmen ein gewisses Qualitätsniveau mit seinen Produkten und Prozessen erreicht.

Der Eindruck, das Projekt zur ein zertifiziertes Unternehmen durchführen zu lassen, wäre teurer trügt hierbei. Tatsächlich sind lediglich die initialen Kosten der Entwicklung höher. Durch diese fallen jedoch die Kosten, welche nachgelagert entstünden geringer und seltener aus. Konkret handelt es sich dabei etwa um Kosten bei einem Softwareversagen, Kosten durch verringerte Arbeitsgeschwindigkeit durch schlechte Nutzererfahrung und geringfügige Fehler oder um Kosten bei der Weiterentwicklung, die mit einem Basisprodukt von hoher Qualität günstiger ist. Ganz allgemein kann mit der Wahl eines zertifizierten Unternehmens als Dienstleister das in Kauf genommene Risiko geschmälert werden. Durch diesen Effekt entfallen etwa Kosten durch den Ausfall der Lieferung, da sich der Dienstleister sich im Nachhinein nicht als unfähig herausstellen wird, das Projekt umzusetzen.