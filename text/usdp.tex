\section{Unified Software Development Process}

Beim diesem Prozess handelt es sich eher um ein \textit{Framework}, dass in verschiedenen Ausprägungen angewandt wird. Dabei ist die Interpretation, der \textit{Rational Unified Process}, der Firma IBM in höherem Maße verbreitet als die anderen. Der Prozess gliedert sich in vier Phasen, welche nacheinander zu bearbeiten sind. Jede Phase wird mit einem erreichten Meilenstein abgeschlossen. 

Die erste Phase \textbf{Inception} dient dem Ziel, den \textit{Lifecycle Object Milestone} zu erreichen. Der Prozess wird hier initialisiert, indem erste Konzepte und eine Produktvision entwickelt werden. Außerdem identifiziert das Projektteam die wesentlichen Risiken, die zu beachten sind.

In der \textbf{Elaboration}-Phase wird am Erreichen des \textit{Lifecycle Architecture Milestone} gearbeitet. Folglich wird ein Architekturprototyp definiert und die Konstruktionsphase geplant. Dazu notwendig ist, dass mindestens 80\% der Anwendungsfälle detailliert ausgearbeitet sind.

Die getroffenen Designentscheidungen kommen nun in der dritten Phase \textbf{Construction} zu tragen, um den \textit{Initial Operational Capability Mildestone} zu erreichen. Dazu findet die Umsetzung und die Testphase gemeinsam mit kontinuierlicher Verbesserung statt.

Die letzte Phase \textbf{Transition} kümmert sich zum Schluss um die Integration der gelieferten Software. Am Ende ist der \textit{Product Release Milestone} erreicht.

Grundsätzlich arbeitet jede Phase iterativ-inkrementell, wie stark kommt jedoch auf die spezielle Phase an. Man spricht außerdem von einem \textit{Use-Case} getriebenen Ansatz, da diese zum Einsatz kommen, um die funktionalen Anforderungen zu definieren. Den stärksten Fokus erhällt jedoch stets die Architektur der Software, um etwaige Risiken zu verringern. Dieser Aspekt ist jedoch auch generell von Bedeutung, weshalb im Zuge der \textit{Fokussierung auf das Risiko} diejenigen Use-Cases zuerst bearbeitet werden, welche das höchste Risiko bergen.