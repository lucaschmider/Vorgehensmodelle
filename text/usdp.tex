\section{Unified Software Development Process}

Beim diesem Prozess handelt es sich eher um ein \textit{Framework}, dass in verschiedenen Ausprägungen angewandt wird. Dabei ist die Interpretation, der \textit{Rational Unified Process}, der Firma IBM in höherem Maße verbreitet als die anderen. Der Prozess gliedert sich in vier Phasen, welche nacheinander zu bearbeiten sind. Jede Phase wird mit einem erreichten Meilenstein abgeschlossen. 

Die erste Phase \textbf{Inception} dient dem Ziel, den \textit{Lifecycle Object Milestone} zu erreichen. Der Prozess wird hier initialisiert, indem erste Konzepte und eine Produktvision entwickelt werden. Außerdem identifiziert das Projektteam die wesentlichen Risiken, die zu beachten sind.

In der \textbf{Elaboration}-Phase wird am Erreichen des \textit{Lifecycle Architecture Milestone} gearbeitet. Folglich wird ein Architekturprototyp definiert und die Konstruktionsphase geplant. Dazu notwendig ist, dass mindestens 80\% der Anwendungsfälle detailliert ausgearbeitet sind.

Die getroffenen Designentscheidungen kommen nun in der dritten Phase \textbf{Construction} zu tragen, um den \textit{Initial Operational Capability Mildestone} zu erreichen. Dazu findet die Umsetzung und die Testphase gemeinsam mit kontinuierlicher Verbesserung statt.

Die letzte Phase \textbf{Transition} kümmert sich zum Schluss um die Integration der gelieferten Software. Am Ende ist der \textit{Product Release Milestone} erreicht.

Grundsätzlich arbeitet jede Phase iterativ-inkrementell, wie stark kommt jedoch auf die spezielle Phase an. Man spricht außerdem von einem \textit{Use-Case} getriebenen Ansatz, da diese zum Einsatz kommen, um die funktionalen Anforderungen zu definieren. Den stärksten Fokus erhällt jedoch stets die Architektur der Software, um etwaige Risiken zu verringern. Dieser Aspekt ist jedoch auch generell von Bedeutung, weshalb im Zuge der \textit{Fokussierung auf das Risiko} diejenigen Use-Cases zuerst bearbeitet werden, welche das höchste Risiko bergen.

\subsection{Einordnung in den Ordnungsrahmen der HELENA-Studie}
Die Studie ordnet die Implementierung von IBM selbst in ihren Rahmen ein. So handelt es sich beim \textit{Rational Unified Process} um eine Methode. Insbesondere der Fakt, dass es sich eher um ein Framework handelt, welches detailliert ausgestaltet werden muss, spricht für diese Aussage. Außerdem wird der gesamte Projektentwicklungszyklus mit dem Modell gestaltet, die Details müssen in den einzelnen Phasen jedoch präziser ausgestaltet werden.

Das Modell ist mit den meisten Clustern kompatibel. Dafür ist es notwendig, dass die im Cluster enthaltenen Ansätze inkrementelles Arbeiten erlauben und sich nach Möglichkeit nicht zu stark auf mehrere Phasen des Entwicklungsprozesses beziehen. Behandeln die enthaltenen Modelle ein Vorgehen, um Arbeit einzuplanen und Anforderungen zu definieren, muss unter Umständen ein Kompromiss geschaffen werden und diese Schritte gebündelt vor der Entwicklung auszuführen.

Das Modell passt lediglich nicht in das Cluster 3, da hier mit dem V-Modell bereits ein ganzheitlicher Ansatz enthalten ist. Das V-Modell sieht außerdem keine inkrementellen Arbeiten zu. Die Ansätze, welche sich in den anderen Clustern befinden, gestalten in der Regel die Phase 2 oder 3 der \textit{RUP}.

\subsection{RUP geht nicht über die Studie hinaus}
Da das Modell selbst Gegenstand der Studie war, geht es nicht über die Studie hinaus. Es ist jedoch anzumerken, dass das Modell eine relativ unbedeutende Rolle in der Studie spielt, da lediglich ein Teilnehmer (1,4\% bei 69 Teilnehmern) dieses Modell verwendet.